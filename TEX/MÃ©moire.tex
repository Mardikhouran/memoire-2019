%!TEX TS-program = xelatex
%!TEX encoding = UTF-8 Unicode
\documentclass[12pt,twoside,a5paper]{memoir}
\chapterstyle{thatcher}
\usepackage{geometry}
\geometry{a5paper}
\usepackage{makeidx}
\usepackage{graphicx}
\usepackage{amssymb}
\usepackage{soul}	%surlignage
\usepackage[main=french,english]{babel}
\usepackage{fontspec,xltxtra,xunicode}
\defaultfontfeatures{Mapping=tex-text}
\usepackage{emptypage}	%pas de numéros sur pages blanches
\usepackage{pdfpages}
\graphicspath{{images/}}

\setromanfont[Ligatures={Required,%
Common,Contextual,Rare,Historic,TeX},Numbers=OldStyle,RawFeature={+ss05,+dlig,+hlig,+calt,+liga},]{EB Garamond}
\setsansfont[Scale=MatchLowercase,Mapping=tex-text]{Noto Sans}
\setmonofont[Scale=MatchLowercase]{Everson Mono}

\sethlcolor{lightgray}
\newcommand{\code}[1]{\hl{\texttt{#1}}}
\usepackage{listings}
	
\usepackage[autostyle=true]{csquotes}
\usepackage[
backend=biber,
style=authoryear,
citestyle=authoryear-icomp,
sorting=nyt,
ibidtracker=true
]{biblatex}
\AtEveryBibitem{
 	\clearlist{language}
	\clearfield{day}%
   	\clearfield{month}%
   	\clearfield{endday}%
   	\clearfield{endmonth}%
	\clearfield{doi}
	\clearfield{issn}
	\clearfield{isbn}
}
\addbibresource{MasterÉd.bib}
\DeclareCiteCommand{\citeyear}
    {}
    {\bibhyperref{\printdate}}
    {\multicitedelim}
    {}
\DeclareCiteCommand{\citeyearpar}
    {}
    {\mkbibparens{\bibhyperref{\printdate}}}
    {\multicitedelim}
    {} %pour les hyperliens
\renewcommand{\mkibid}[1]{\emph{#1}}

\usepackage{rotating,booktabs}	%tableaux horizontaux
\usepackage[flushleft]{threeparttable}
\usepackage{multicol}
\usepackage{longtable}		%plus sûr que tabu ?
\usepackage{diagbox}
\def\@xobeysp{ } %espaces sécables (table des matières)
\usepackage[hidelinks]{hyperref}
\usepackage[toc,nonumberlist,acronym,nomain]{glossaries}
%\newglossaryentry{a_posteriori}{}
%\newglossaryentry{a_priori}{}
%\newglossaryentry{artistique}{}
%\newglossaryentry{auxiliaire}{}
%\newglossaryentry{conlang}{}
%\newglossaryentry{conlanger}{}
%\newglossaryentry{construite}{}
%\newglossaryentry{diégèse}{}
%\newglossaryentry{fictionnelle}{}
%\newglossaryentry{glossopoète}{}
%\newglossaryentry{idéolangue}{}
%\newglossaryentry{idéolinguiste}{}
%\newglossaryentry{interlangue}{}
%\newglossaryentry{naturaliste}{}
%\newglossaryentry{naturelle}{}
%\newglossaryentry{pasigraphie}{}
\newglossaryentry{casseau}{
name=casseau,
description={élément décoratif du texte}
}
\makeglossaries %faire avec terminal (pas de chemin absolu > cd répertoire avant)
\usepackage{glossary-mcols}
\renewcommand*{\glsmcols}{2}
\setglossarystyle{mcolindex}
\usepackage{arabxetex} %mettre à la fin, mais avant les redefs de sections
\newfontfamily\arabicfont[Script=Arabic]{Times New Roman}
\usepackage{chngcntr}		%notes de bas de page
\counterwithout{footnote}{chapter}
\usepackage[restart,breakwithin]{parnotes}	%notes de chapitres
\renewcommand{\parnotefmt}[1]{\normalsize\rmfamily\noindent #1}

\usepackage{setspace}	%contrôle des espacements
\setsecnumdepth{subsection}
\settocdepth{subsection}
\renewcommand{\thechapter}{\Roman{chapter}}
\renewcommand{\thesection}{\arabic{section}}
\renewcommand{\thesubsection}{\arabic{section}.\roman{subsection}}
\setsecheadstyle{\large\bfseries}		%
\setsubsecheadstyle{\normalsize\bfseries\textit}		%subsections en italiques et gras
\setsubsubsecheadstyle{\small\bfseries}		%
\usepackage[inline]{enumitem}	%change les compteurs de listes
\usepackage{numprint} %apparence des chiffres

\expandafter\def\expandafter\quote\expandafter{\quote\singlespacing}	%espacement citations

\title{Rééditer un texte philosophico-linguistique : la digitalisation de \emph{An Essay towards a Real Character, and a Philosophical Language} de John Wilkins (1668)}
\author{Alexis Huchelmann}
\date{}

\begin{document}
\frontmatter
\maketitle
\pagenumbering{gobble}
%\include{chapitres/remerciements}
\mainmatter
\chapter*{Introduction}\label{introduction}
\addcontentsline{toc}{chapter}{\nameref{introduction}}

Comment rendre la littérature d'hier plus accessible ?

Il ne s'agit pas de multiplier des duplicatas de vieux textes et de les distribuer à tous vents.
Lorsque l'on parle d'accessibilité, la qualité importe autant que la quantité.
Ce qui était lisible hier ne l'est plus forcément aujourd'hui : sans parler des dégradations techniques qu'ont pu subir les supports, il faut prendre en compte les différences langagières entre les époques, et les différences entre les conventions typographiques (ainsi les abréviations).
Il importe de capturer avant tout le sens de l'original avant son apparence.
À cette fin, on transcrit le texte en caractères informatiques standards ; et pour encoder encore plus d'information, on l'organise dans un schéma standardisé.

Ranger les choses dans un schéma était aussi le but de John Wilkins dans \emph{An Essay Towards a Real Character, and a Philosophical Language} en 1668.
Dans cet essai, l'évêque de Chester proposa à ses pairs de la Royal Society une nouvelle langue basée sur une hiérarchie rationnelle du monde, en accord avec les développements de la science de l'époque.
Le livre comprend plus de quarante tableaux hiérarchiques de mots classés par sens, un thésaurus regroupant synonymes et paronymes de la langue anglaise de l'époque et une proposition d'écriture idéographique.
Un simple duplicata ne rendrait pas justice à la masse d'information présentée.
\listoffigures
\clearpage
\setcounter{part}{0}
\chapter{Nature et histoire de la digitalisation}\label{HistDigit}
%\chapterprecis{Où l'on pose les bases théoriques sur lesquelles repose l'entreprise considérée dans le présent ouvrage.}
\section{Définition}\label{HistDigitDef}
\subsection{Digitalisation vs numérisation}
La digitalisation est à distinguer de la numérisation.
Dans les deux cas, il y a création d'un fichier électronique, mais l'information n'est pas la même : la numérisation est une photographie du texte, la digitalisation convertit l'information de l'original dans un format conservant l'information sémantique : quels sont les caractères composant le texte, quelle est sa hiérarchie interne.
\subsection{Différentes encodages}
En passant de l'imprimé au numérique, il s'agit de convertir les caractères du texte en entités lisibles par la machine à des fins de manipulation.
Aux débuts de l'ère informatique, 
\section{Emploi}\label{HistDigitEmp}
\subsection{Pour la sauvegarde du patrimoine littéraire}
\subsubsection{Projet Gutenberg}
Le 4 juillet 1971, Michael Hart pose la première pierre du projet Gutenberg en mettant en ligne la Déclaration de l'indépendance des États-Unis depuis les ordinateurs de l'université de l'Illinois.
À l'époque, les réseaux informatiques sont limités aux grandes universités américaines, et la saisie et la diffusion des textes sont relativement lentes ; mais Hart est vite rejoint par des volontaires, et le rythme de publication augmente d'année en année.
De dix textes en 1989, on passe à mille en 1997, puis à dix mille dès 2003 \parencite{lebert_projet_2010}.

Comme tous les textes sont mis à disposition librement sur des serveurs situés aux États-Unis, ils doivent respecter les lois américaines sur le copyright.
Du fait de la diffusion mondiale du projet, il peut entrer en conflit avec les lois locales sur le droit d'auteur : ainsi, en 2015, S. Fischer Verlag, un éditeur allemand, demanda le retrait de dix-huit titres de la bibliothèque en ligne que les lois locales protégeaient encore.
Un compromis fut trouvé en 2018 d'après lequel la fondation Gutenberg supprima l'accès à toutes les pages de son domaine aux adresses IP allemandes \parencite{noauthor_court_2018}.
\subsubsection{Google Books}
En 2002, Google commence à scanner 
\subsubsection{Internet Archive}
L'organisation à but non-lucrative Internet Archive, surtout connue pour son travail de sauvegarde des sites web (Wayback Machine) présente également un volet de stockage de numérisations, dont des livres.
En octobre 2019, près de 2500000 livres étaient présents sur le site, une moitié librement consultable par tous (livres hors-copyright), l'autre pouvant être lue en ligne par les inscrits au service (une personne à la fois peut consulter un livre\footnote{\url{https://archive.org/details/books} (consulté le 15 octobre 2019)}.

Internet Archive propose un service payant de numérisation des collections des institutions.
Avec 33 centre de scanneurs répartis dans le monde entier, l'organisation peut numériser mille pages par jour, qui sont transformées en PDF couleur de qualité haute permettant la recherche de texte (grâce à la reconnaissance optique des caractères)\footnote{\url{https://archive.org/scanning} (consulté le 15 octobre 2019)}.
Elle permet aussi aux utilisateurs enregistrés de téléverser leurs propres numérisations dans une collection open source \og Community Texts \fg\footnote{\url{https://archive.org/details/opensource} (consulté le 15 octobre 2019)}.
\subsubsection{Wikisource}
La fondation Wikimedia, qui chapeaute entre autres l'encyclopédie Wikipédia, lança en 2003 Wikisource pour stocker des textes libres de droits (domaine public ou œuvres libres).
Ils sont digitalisés, c'est à dire que le texte brut est mis en forme selon la syntaxe wiki.
Comme Wikisource est un projet collaboratif ouvert, il se pose la question de la validité des sources ; une extension du logiciel de Wikimedia permet de comparer le texte à une image numérisée servant de source\footnote{\url{https://en.wikipedia.org/wiki/Wikisource} (consulté le 15 octobre 2019}.
\subsubsection{Initiatives nationales}
À côté de ces projets d'envergure mondiale (quoique basés aux USA), certains pays fournissent leurs propres efforts pour numériser leurs fonds patrimoniaux.
Ainsi, la Bibliothèque nationale de France gère depuis 1997 le portail Gallica, mettant en ligne les images des livres des collections nationales accompagnés d'une notice descriptive\footnote{\url{https://gallica.bnf.fr}}.
La plateforme Europeana lancé en 2008 par la Commission européenne sert d'interface de recherche permettant d'accéder aux fonds numériques des institutions culturelles européennes partenaires, dont Gallica\footnote{\url{https://www.europeana.eu/portal/fr}}.
\subsection{Dans les humanités}
La digitalisation est très importante dans les humanités, où elle permet de sauvegarder des textes dont le support a vieilli, et de fournir un matériel de travail interactif aux chercheurs.

\chapter{Le schéma d'encodage TEI}\label{TEI}
\section{Un standard SGML et XML}\label{TEIXML}
Le SGML (\emph{Standard Generalized Markup Language}) est un langage de description à balises apparu en 1986 qui permet de définir des éléments et leurs relations hiérarchiques les uns aux autres.
Dès 1987, un groupe de chercheurs new-yorkais utilise ce standard comme base sur laquelle créer des éléments de descriptions de toutes sortes de textes : manuscrits, dictionnaires, pièces de théâtres, lettres, etc, de toutes époques et de toutes langues.
Ils publient leurs consignes en 1990 en tant que TEI P1 (Text Encoding Initiative Proposal 1), et la P3 publié en 1994 contient plus de 600 éléments définis sur 1292 pages \parencite{vanhoutte_introduction_2004}.

Créé en 1998, le XML (\emph{eXtented Markup Language}) est un dérivé de SGML à la syntaxe plus versatile.
Moins d'éléments sont définis à la base, et il exige un encodage Unicode là où le SGML laissait les utilisateurs choisir leur jeu de caractères.
TEI P4, publié en 2002, est compatible avec cette nouvelle spécification autant qu'avec le SGML ; mais en 2007, la version P5 est exclusivement taillée pour le XML.
Elle a des exigences plus strictes quand au type de contenus que peuvent présenter les attributs et les balises, introduit de nouveaux éléments permettant de décrire en détail les entités nommées par des noms propres (personnes, lieux, peuples, etc.), et s'ouvre davantage à la customisation en améliorant la modularité des éléments \parencite{wittern_making_2009}.
En date du 16 juillet 2019, les indications TEI P5 sont décrites sur 1934 pages.
\section{TEI Lite}\label{TEILite}
La spécification TEI complète n'est jamais employée dans son entièreté dans un projet.
Des sous-spécifications moins exhaustives, avec quelquefois des ajouts modulables, sont destinées à des projets à l'envergure bien définie, comme la description des documents épigraphiques (EpiDoc), la mise en forme d'articles de journal (jTEI Article), ou la description de corpus linguistiques (Corpus)\footnote{\url{https://tei-c.org/guidelines/customization/}}.
L'une d'entre elle est décrite comme pouvant \og couvrir 90\% des besoins de 90\% de la communauté\fg : il s'agit de TEI Lite\footnote{\url{https://tei-c.org/guidelines/customization/Lite/}}.
Comme le laisse entendre son slogan, TEI Lite est souvent la seule version du schéma que rencontre les utilisateurs, sa description ne fait que 294 pages.
\section{Potentiel descriptif}\label{TEIPot}
Un document TEI, compris entre les balises \code{<TEI>…</TEI>} a deux éléments principaux : l'en-tête \code{<teiHeader>} et le texte \code{<text>}.
%\begin{lstlisting}[language=XML,label=TEIminim,frame=single,caption=Some Code]
%\end{lstlisting}
\chapter{Le livre de John Wilkins}\label{Wilkins}
%\chapterprecis{Où l'on rencontre le personnage central de notre récit, les circonstances de sa naissance et sa destinée dans le monde.}
\section{L'auteur}\label{WilkinsAut}
John Wilkins (1614-1672) était un ecclésiastique anglican et scientifique anglais aux intérêts éclectiques, un des fondateurs de la Royal Society.
\subsection{Biographie}
Né en 1614 dans une famille d'artisans dans le Northamptonshire, Wilkins fut éduqué à Oxford d'où il sorti avec une maîtrise en 1634.
En 1637, il entra dans les ordres dans l'Église d'Angleterre, d'abord comme curé dans son village natal de Fawsley, puis réussit à obtenir la place de chapelain auprès de William Fiennes, vicomte de Say and Seale.
Lors de l'éclatement de la première révolution anglaise en 1642, Wilkins se réfugia à Londres où il devint chapelain d'abord chez Lord Berkeley, puis chez le prince-électeur palatin Charles Louis, neveu du roi Charles I\ier.
C'est là qu'il participa à l'activité intellectuelle d'un cercle de savants oxfordiens, en exil dans la capitale après la prise de la ville par les forces parlementaires en 1646 ; on y trouvait entre autres les médecins Charles Scarborough et William Harvey, l'astronome Samuel Foster, et le mathématicien John Wallis.
Ce club informel fut la base de ce qui allait devenir, lors du retour à Oxford de ces savants en 1648, la Royal Society, constituée à Wadham College dont Wilkins était devenu le directeur.
En 1659, Wilkins devient directeur de Trinity College à Cambridge, d'après les vœux de Richard Cromwell, le Lord Protecteur, dont il avait épousé la tante Robina en 1656.
Mais il ne gardera ce poste que dix mois, puisqu'en 1660 Charles II reprend le pouvoir en Angleterre et les ex-partisans de Cromwell perdent les faveurs des autorités.
Wilkins revint vers Londres, et occupa une variété de postes ecclésiastiques : prêcheur à Gray's Inn, vicaire de St Lawrence Jerry, doyen de Ripon, tout en occupant un poste de secrétaire à la Royal Society.
En 1668 enfin, il retrouva la pleine faveur du roi et se vit offrir l'évêché de Chester ; quatre ans plus tard, en 1672, il mourut, et fut enterré à St Lawrence Jerry.
\subsection{Autres écrits}
Le premier livre de John Wilkins fut un traité d'astronomie titré \emph{The Discovery of a World in the Moone} en 1638.

En 1648, il publia un petit traité de cryptographie intitulé \emph{Mercurie; or the Secret and Swift Messenger}, ainsi que \emph{Mathematical Magick} qui introduit le lecteur aux sciences mécaniques.
\section{L'essai}\label{WilkinsEssai}
\subsection{Contenu}
L'imprimé contient deux livres, le premier étant l'essai à proprement parler, et le second étant le dictionnaire faisant correspondre les mots de la langue anglaise aux tables philosophiques développées précédemment.

Le premier livre est divisé en quatre parties : la première, le prologue, présente une réflexion linguistique et philosophique sur l'état des langues naturelles contemporaines, leurs origines et leurs évolutions (appelées \og changements et corruptions\fg), l'usage des différents systèmes d'écriture et leurs défauts.
Wilkins propose devant ces imperfections de bâtir une nouvelle langue sur des bases rationnelles et philosophiques, en commençant par définir quels concepts et objets sont à nommer.

Les chapitres de la deuxième partie présentent les quarante \emph{genres} ou têtes sémantiques, parmi lesquelles six forment des catégories à part (\og notions transcendantales générales\fg, \og relations transcendantales mixtes\fg, \og relations transcendantales d'action\fg, \og la communication\fg, \og le Créateur\fg, et \og le monde\fg) et le reste est divisé en cinq familles (\og substance\fg, \og quantité\fg, \og qualité\fg, \og action\fg, et \og relation\fg).
Ces genres comprennent par exemple \og les pierres \fg, \og les oiseaux \fg, \og les mesures \fg, ou \og les titres ecclésiastiques \fg et sont divisés à leur tour en six \emph{différences} chacun, elles-mêmes contenant plusieurs espèces.

La troisième partie de l'essai concerne la grammaire, que Wilkins sépare en deux, la grammaire \emph{naturelle} ou universelle concernant toutes les langues, et la grammaire \emph{instituée} ou \emph{particulière} propre à chacune.
Les différentes parties du discours de la grammaire classique sont passées en revue, ainsi que les moyens de les dériver ; puis on se penche sur la phonétique ou \emph{orthographe} comme l'entend Wilkins.
Un schéma de transcription phonétique est produit, exemplifié par les versions anglaises du Notre-Père et du Credo.

Enfin, la quatrième partie

\subsection{Genèse et rédaction}
Le \textsc{xvii}\ieme{} siècle en Europe voit des savants se pencher sur la question de la langue.

\subsection{Réception}
\section{Littérature scientifique}\label{WilkinsLitt}
George Edmonds

Borges

Okrent
\section{Choix de l'ouvrage}\label{WilkinsChoix}
\subsection{Intérêt du sujet}
L'intérêt porté par le grand public aux langues construites changea de force et de nature au cours des siècles : de la quête du langage originel, en passant par la création de langues plus rationnelles et la recherche d'un instrument de communication international, on assiste aujourd'hui à la vogue des langues fictionnelles dans les films et séries télévisées \parencite{huchelmann_les_2019}.

À côté de cela, une communauté d'inventeurs de langues s'est retrouvée sur Internet et produit des échanges en nombre toujours croissant autour de leurs propres créations et celles des temps passés.
Leurs efforts sont en majorité orientés vers leur propre plaisir, et ils suivent avec intérêt les travaux des autres.
Bien qu'ils aient conscience des réalisations de leurs prédécesseurs avant l'âge digital, seules les plus connues d'entre elles leur sont accessibles dans une version électronique : c'est le cas par exemple de l'espéranto\footnote{\url{esperanto.net} (consulté le 19 octobre 2019)} et du volapük\footnote{\url{volapük.com} (consulté le 19 octobre 2019)}, deux langues à vocation international dont la communauté de locuteur désire la dissémination, ou des langues elfiques de J.R.R. Tolkien.
Les projets n'ayant pas eu autant de succès sont parfois, s'ils ont été publiés en tant que livres, numérisés dans diverses collections.



\subsection{Précédentes numérisations et digitalisations}\label{WilkinsChoixNum}
\subsubsection{Numérisation}
Il existe plusieurs numérisations depuis Google Books ; seules trois peuvent être lues intégralement, provenant d'originaux de la bibliothèque nationale de la République Tchèque\footnote{\url{https://books.google.fr/books?id=Bu7pwpr5qBcC} (consulté le 16 octobre 2019)}, de la bibliothèque municipale de Lyon\footnote{\url{https://books.google.fr/books?id=BCCtZjBtiEYC} (consulté le 16 octobre 2019)} et de la bibliothèque de l'État de Bavière\footnote{\url{https://books.google.fr/books?id=Q85TAAAAcAAJ} (consulté le 16 octobre 2019)}.
Elles permettent la recherche de texte, mais la reconnaissance automatique des caractères est très mal adaptée à la police et aux ligatures employées ici, et ne permet de rien extraire qui soit utilisable.
Lues en ligne, ces images sont en couleur ; une fois téléchargées en tant que PDF, elles sont en noir et blanc, ce qui baisse la qualité de la reproduction.
\subsubsection{Digitalisation de la Text Creation Partnership}
La Text Creation Partnership est une initiative conjointe de l'Université du Michigan, de la bibliothèque Bodléienne d'Oxford, le portail en ligne ProQuest et le Council on Library and Information Resources pour digitaliser leurs ouvrages déjà numérisés selon un même standard et mettre ces ressources en ligne.
Les trois collections disponibles sont Early English Books Online (1473-1700), Eighteenth-Century Collections Online et Early American Imprints.
Plutôt que d'employer la reconnaissance optique des caractères, l'initiative employa une centaine de personnes pour encoder les textes manuellement dans un schéma SGML dérivé de TEI P3, qui fut plus tard converti en XML dérivé de TEI P4\footnote{\url{https://textcreationpartnership.org/about-the-tcp/} (consulté le 17 octobre 2019)}.

\emph{An Essay towards a Real Character} fut saisi en 2004 par Mona Logarbo.
Il est divisé en chapitres et l'organisation interne du texte est assez fidèlement respectée, par exemple les listes hiérarchiques d'éléments apparaissent sous formes de listes HTML imbriquées, les tableaux apparaissent en tant que tel, et les annotations marginales sont intégrées dans le texte comme appels de note.
Les changements de pages et de lignes dans l'édition originale sont toujours indiqués.
Cependant, les images ne sont pas incluses, pas plus que les caractères non-latins.

\section{Fac-similé ou réédition ?}\label{WilkinsReed}
Une digitalisation peut consister en une copie du texte de base, où le nombre et l'organisation des pages restent les mêmes, et  le texte transcrit le plus fidèlement possible.
\subsection{Apporter quelque chose de neuf}
À la digitalisation du Text Encoding Project, il y a plusieurs choses que l'on pourrait ajouter :
\begin{enumerate}
\item Une meilleure gestion des caractères non-latins, dont l'écriture philosophique.
\item Des liens externes vers les ouvrages cités dans le texte.
\item Des conversions dynamiques vers le mot en langue philosophique depuis les éléments du thésaurus.
\end{enumerate}
\subsection{Les limites de la réédition}

\include{chapitres/Défis}
\include{chapitres/Méthodologie}
\chapter*{Conclusion}\label{Conclusion}
La réalisation d'un ouvrage digital est compliquée.
\nocite{*}

\backmatter
\printindex
\printbibliography[heading=bibintoc]
\pagebreak
\tableofcontents*
\end{document}