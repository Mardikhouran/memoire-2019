\chapter*{Introduction}\label{introduction}
\addcontentsline{toc}{chapter}{\nameref{introduction}}

Comment rendre la littérature d'hier plus accessible ?

Il ne s'agit pas de multiplier des duplicatas de vieux textes et de les distribuer à tous vents.
Lorsque l'on parle d'accessibilité, la qualité importe autant que la quantité.
Ce qui était lisible hier ne l'est plus forcément aujourd'hui : sans parler des dégradations techniques qu'ont pu subir les supports, il faut prendre en compte les différences langagières entre les époques, et les différences entre les conventions typographiques (ainsi les abréviations).
Il importe de capturer avant tout le sens de l'original avant son apparence.
À cette fin, on transcrit le texte en caractères informatiques standards ; et pour encoder encore plus d'information, on l'organise dans un schéma standardisé.

Ranger les choses dans un schéma était aussi le but de John Wilkins dans \emph{An Essay Towards a Real Character, and a Philosophical Language} en 1668.
Dans cet essai, l'évêque de Chester proposa à ses pairs de la Royal Society une nouvelle langue basée sur une hiérarchie rationnelle du monde, en accord avec les développements de la science de l'époque.
Le livre comprend plus de quarante tableaux hiérarchiques de mots classés par sens, un thésaurus regroupant synonymes et paronymes de la langue anglaise de l'époque et une proposition d'écriture idéographique.
Un simple duplicata ne rendrait pas justice à la masse d'information présentée.