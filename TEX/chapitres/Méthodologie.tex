\chapter{Méthodologie de travail}\label{Method}
%\chapterprecis{Où l'on explique comment sera réalisée l'entreprise considérée présentemment.}
\section{Choix des sources}\label{MethodSource}
Pour réaliser le livre électronique, on prendra pour base la digitalisation préexistante du Texte Encoding Project, en se référant aux numérisations Google Books pour vérification, et enfin à un exemplaire imprimé original.
\subsection{Sources physiques}
Il en existe trois exemplaires physiques du livre à la Bibliothèque Nationale Universitaire de Strasbourg : le texte original imprimé en 1668, qui est uniquement consultable sur place ; et deux fac-similés produits par le service d'impression à la demande à partir de numérisations.
\subsection{Sources numériques : les images}
Il existe plusieurs numérisations depuis Google Books ; seules trois peuvent être lues intégralement, provenant d'originaux de la bibliothèque nationale de la République Tchèque\footnote{\url{https://books.google.fr/books?id=Bu7pwpr5qBcC} (consulté le 16 octobre 2019)}, de la bibliothèque municipale de Lyon\footnote{\url{https://books.google.fr/books?id=BCCtZjBtiEYC} (consulté le 16 octobre 2019)} et de la bibliothèque de l'État de Bavière\footnote{\url{https://books.google.fr/books?id=Q85TAAAAcAAJ} (consulté le 16 octobre 2019)}.
Elles permettent la recherche de texte, mais la reconnaissance automatique des caractères est très mal adaptée à la police et aux ligatures employées ici, et ne permet de rien extraire qui soit utilisable.
Lues en ligne, ces images sont en couleur ; une fois téléchargées en tant que PDF, elles sont en noir et blanc, ce qui baisse la qualité de la reproduction.
\subsection{Sources numériques : l'Université du Michigan}
La digitalisation d'ETRC par le Text Encoding Project de l'université du Michigan (cf \ref{WilkinsChoixNum}) a été saisie originellement dans un schéma SGML ressemblant à la spécification TEI 3, converti par la suite en un schéma XML proche de la spécification TEI 4.
À partir de là, on recherchera les différences entre ce schéma et le schéma cible (TEI 5) et on établira un tableau de correspondances qui servira à la conversion automatique de l'un à l'autre grâce à une feuille de style XSLT (voir section \ref{MethodConv}).
\section{Travail éditorial : recherches à faire et décisions à prendre}\label{MethodTrav}
\subsection{Omissions dans la digitalisation du Text Encoding Project}
En plus de la non prise en compte des caractères non-latins, 
\subsection{Errata}
Wilkins donne une page d'errata en début du livre, signalant les erreurs découvertes après l'impression.
La question est de savoir si l'on conserve cette page telle que dans la source, ou si l'on intègre les corrections dans le corps du texte.
L'élément \code{<choice>} de la spécification TEI permet de signaler une modification éditorial dans le texte, en encadrant un élément \code{<sic>} correspondant au texte (fautif) original, et un élément \code{<corr>} correspondant à la correction.
Ce dernier peut également préciser l'origine de la correction avec un attribut \code{@resp}, renvoyant à un élément \code{respStmt} définissant qui a pris cette décision de correction.
On peut définir plusieurs sources de correction ainsi.
\subsection{Particularités orthographiques et typographiques}
Wilkins n'emploie pas les conventions typographiques actuelles.
Quelles sont celles qu'une réédition doit conserver, si le but est de respecter la source tout en étant lisible par les lecteurs d'aujourd'hui ?
\subsubsection{Capitalisation}
Certains mots, en majorité des substantifs et adjectifs commencent par une majuscule à l'intérieur des phrases, comme on peut le voir par exemple en p. 353 :
\begin{quotation}
\begin{otherlanguage*}{english}
Noun \emph{Adjectives} need not have any note to expreſs \emph{Number}, \emph{Gender}, \emph{Caſe}, becauſe in all theſe they agree with their Subſtantives; unleſs ſuch Adjectives as are uſed Subſtantively, by reaſon of their compoſition with the Tranſcendental marks of \emph{Perſon}, \emph{Thing}, \emph{Time}, \emph{Place}, \&c.
\end{otherlanguage*}
\end{quotation}
Comme cela n'entraîne pas de problèmes de lecture, on les conservera telles quelles.
\subsubsection{s long 〈ſ〉}
ETRC emploie deux caractères là où l'anglais moderne n'utilise que le 〈s〉 minuscule : le second est le s long 〈ſ〉, apparaissant dans tous les contextes excepté à la fin des mots.
Cette lettre a cessé d'être d'un usage courant vers le début du \textsc{xix}\ieme{} siècle, pour des raison d'économie auprès des imprimeurs, et parce qu'elle pouvait se confondre à la lecture avec un 〈f〉 \parencite{west_rules_2006}.
Pour cette dernière raison, il semble préférable d'utiliser le 〈s〉 rond par défaut ; mais comme certaines polices informatiques, comme EB Garamond, proposent des variantes de caractères dont le s long, que l'on peut activer à loisir.
\subsubsection{Orthographe alternatives}

\subsubsection{Réclame}
Le premier mot de chaque page est répété au bas de la page précédente ; il s'agissait d'une pratique indiquant au relieur quelles pages devait se suivre \parencite{roberts_catchword_1982}.
Cette particularité purement technique n'a pas besoin d'être répétée dans une édition électronique.
\section{Conversion dans les formats HTML et EPUB}\label{MethodConv}
À partir du fichier encodé en XML, il est possible de générer automatiquement un fichier HTML, et à partir de celui-ci un fichier EPUB.
Pour cela, on emploiera une feuille de style XSLT.
Celle-ci permet de transformer un fichier XML en manipulant les balises et les attributs.
Sa cible peut être un nouveau fichier XML, un fichier HTML, ou même un fichier \LaTeX{} qui servira de base à un PDF.