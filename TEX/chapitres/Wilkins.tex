\chapter{Le livre de John Wilkins}\label{Wilkins}
%\chapterprecis{Où l'on rencontre le personnage central de notre récit, les circonstances de sa naissance et sa destinée dans le monde.}
\section{L'auteur}\label{WilkinsAut}
John Wilkins (1614-1672) était un ecclésiastique anglican et scientifique anglais aux intérêts éclectiques, un des fondateurs de la Royal Society.
\subsection{Biographie}
Né en 1614 dans une famille d'artisans dans le Northamptonshire, Wilkins fut éduqué à Oxford d'où il sorti avec une maîtrise en 1634.
En 1637, il entra dans les ordres dans l'Église d'Angleterre, d'abord comme curé dans son village natal de Fawsley, puis réussit à obtenir la place de chapelain auprès de William Fiennes, vicomte de Say and Seale.
Lors de l'éclatement de la première révolution anglaise en 1642, Wilkins se réfugia à Londres où il devint chapelain d'abord chez Lord Berkeley, puis chez le prince-électeur palatin Charles Louis, neveu du roi Charles I\ier.
C'est là qu'il participa à l'activité intellectuelle d'un cercle de savants oxfordiens, en exil dans la capitale après la prise de la ville par les forces parlementaires en 1646 ; on y trouvait entre autres les médecins Charles Scarborough et William Harvey, l'astronome Samuel Foster, et le mathématicien John Wallis.
Ce club informel fut la base de ce qui allait devenir, lors du retour à Oxford de ces savants en 1648, la Royal Society, constituée à Wadham College dont Wilkins était devenu le directeur.
En 1659, Wilkins devient directeur de Trinity College à Cambridge, d'après les vœux de Richard Cromwell, le Lord Protecteur, dont il avait épousé la tante Robina en 1656.
Mais il ne gardera ce poste que dix mois, puisqu'en 1660 Charles II reprend le pouvoir en Angleterre et les ex-partisans de Cromwell perdent les faveurs des autorités.
Wilkins revint vers Londres, et occupa une variété de postes ecclésiastiques : prêcheur à Gray's Inn, vicaire de St Lawrence Jerry, doyen de Ripon, tout en occupant un poste de secrétaire à la Royal Society.
En 1668 enfin, il retrouva la pleine faveur du roi et se vit offrir l'évêché de Chester ; quatre ans plus tard, en 1672, il mourut, et fut enterré à St Lawrence Jerry.
\subsection{Autres écrits}
Le premier livre de John Wilkins fut un traité d'astronomie titré \emph{The Discovery of a World in the Moone} en 1638.

En 1648, il publia un petit traité de cryptographie intitulé \emph{Mercurie; or the Secret and Swift Messenger}, ainsi que \emph{Mathematical Magick} qui introduit le lecteur aux sciences mécaniques.
\section{L'essai}\label{WilkinsEssai}
\subsection{Contenu}
L'imprimé contient deux livres, le premier étant l'essai à proprement parler, et le second étant le dictionnaire faisant correspondre les mots de la langue anglaise aux tables philosophiques développées précédemment.

Le premier livre est divisé en quatre parties : la première, le prologue, présente une réflexion linguistique et philosophique sur l'état des langues naturelles contemporaines, leurs origines et leurs évolutions (appelées \og changements et corruptions\fg), l'usage des différents systèmes d'écriture et leurs défauts.
Wilkins propose devant ces imperfections de bâtir une nouvelle langue sur des bases rationnelles et philosophiques, en commençant par définir quels concepts et objets sont à nommer.

Les chapitres de la deuxième partie présentent les quarante \emph{genres} ou têtes sémantiques, parmi lesquelles six forment des catégories à part (\og notions transcendantales générales\fg, \og relations transcendantales mixtes\fg, \og relations transcendantales d'action\fg, \og la communication\fg, \og le Créateur\fg, et \og le monde\fg) et le reste est divisé en cinq familles (\og substance\fg, \og quantité\fg, \og qualité\fg, \og action\fg, et \og relation\fg).
Ces genres comprennent par exemple \og les pierres \fg, \og les oiseaux \fg, \og les mesures \fg, ou \og les titres ecclésiastiques \fg et sont divisés à leur tour en six \emph{différences} chacun, elles-mêmes contenant plusieurs espèces.

La troisième partie de l'essai concerne la grammaire, que Wilkins sépare en deux, la grammaire \emph{naturelle} ou universelle concernant toutes les langues, et la grammaire \emph{instituée} ou \emph{particulière} propre à chacune.
Les différentes parties du discours de la grammaire classique sont passées en revue, ainsi que les moyens de les dériver ; puis on se penche sur la phonétique ou \emph{orthographe} comme l'entend Wilkins.
Un schéma de transcription phonétique est produit, exemplifié par les versions anglaises du Notre-Père et du Credo.

Enfin, la quatrième partie

\subsection{Genèse et rédaction}
Le \textsc{xvii}\ieme{} siècle en Europe voit des savants se pencher sur la question de la langue.

\subsection{Réception}
\section{Littérature scientifique}\label{WilkinsLitt}
George Edmonds

Borges

Okrent
\section{Choix de l'ouvrage}\label{WilkinsChoix}
\subsection{Intérêt du sujet}
L'intérêt porté par le grand public aux langues construites changea de force et de nature au cours des siècles : de la quête du langage originel, en passant par la création de langues plus rationnelles et la recherche d'un instrument de communication international, on assiste aujourd'hui à la vogue des langues fictionnelles dans les films et séries télévisées \parencite{huchelmann_les_2019}.

À côté de cela, une communauté d'inventeurs de langues s'est retrouvée sur Internet et produit des échanges en nombre toujours croissant autour de leurs propres créations et celles des temps passés.
Leurs efforts sont en majorité orientés vers leur propre plaisir, et ils suivent avec intérêt les travaux des autres.
Bien qu'ils aient conscience des réalisations de leurs prédécesseurs avant l'âge digital, seules les plus connues d'entre elles leur sont accessibles dans une version électronique : c'est le cas par exemple de l'espéranto\footnote{\url{esperanto.net} (consulté le 19 octobre 2019)} et du volapük\footnote{\url{volapük.com} (consulté le 19 octobre 2019)}, deux langues à vocation international dont la communauté de locuteur désire la dissémination, ou des langues elfiques de J.R.R. Tolkien.
Les projets n'ayant pas eu autant de succès sont parfois, s'ils ont été publiés en tant que livres, numérisés dans diverses collections.



\subsection{Précédentes numérisations et digitalisations}\label{WilkinsChoixNum}
\subsubsection{Numérisation}
Il existe plusieurs numérisations depuis Google Books ; seules trois peuvent être lues intégralement, provenant d'originaux de la bibliothèque nationale de la République Tchèque\footnote{\url{https://books.google.fr/books?id=Bu7pwpr5qBcC} (consulté le 16 octobre 2019)}, de la bibliothèque municipale de Lyon\footnote{\url{https://books.google.fr/books?id=BCCtZjBtiEYC} (consulté le 16 octobre 2019)} et de la bibliothèque de l'État de Bavière\footnote{\url{https://books.google.fr/books?id=Q85TAAAAcAAJ} (consulté le 16 octobre 2019)}.
Elles permettent la recherche de texte, mais la reconnaissance automatique des caractères est très mal adaptée à la police et aux ligatures employées ici, et ne permet de rien extraire qui soit utilisable.
Lues en ligne, ces images sont en couleur ; une fois téléchargées en tant que PDF, elles sont en noir et blanc, ce qui baisse la qualité de la reproduction.
\subsubsection{Digitalisation de la Text Creation Partnership}
La Text Creation Partnership est une initiative conjointe de l'Université du Michigan, de la bibliothèque Bodléienne d'Oxford, le portail en ligne ProQuest et le Council on Library and Information Resources pour digitaliser leurs ouvrages déjà numérisés selon un même standard et mettre ces ressources en ligne.
Les trois collections disponibles sont Early English Books Online (1473-1700), Eighteenth-Century Collections Online et Early American Imprints.
Plutôt que d'employer la reconnaissance optique des caractères, l'initiative employa une centaine de personnes pour encoder les textes manuellement dans un schéma SGML dérivé de TEI P3, qui fut plus tard converti en XML dérivé de TEI P4\footnote{\url{https://textcreationpartnership.org/about-the-tcp/} (consulté le 17 octobre 2019)}.

\emph{An Essay towards a Real Character} fut saisi en 2004 par Mona Logarbo.
Il est divisé en chapitres et l'organisation interne du texte est assez fidèlement respectée, par exemple les listes hiérarchiques d'éléments apparaissent sous formes de listes HTML imbriquées, les tableaux apparaissent en tant que tel, et les annotations marginales sont intégrées dans le texte comme appels de note.
Les changements de pages et de lignes dans l'édition originale sont toujours indiqués.
Cependant, les images ne sont pas incluses, pas plus que les caractères non-latins.

\section{Fac-similé ou réédition ?}\label{WilkinsReed}
Une digitalisation peut consister en une copie du texte de base, où le nombre et l'organisation des pages restent les mêmes, et  le texte transcrit le plus fidèlement possible.
\subsection{Apporter quelque chose de neuf}
À la digitalisation du Text Encoding Project, il y a plusieurs choses que l'on pourrait ajouter :
\begin{enumerate}
\item Une meilleure gestion des caractères non-latins, dont l'écriture philosophique.
\item Des liens externes vers les ouvrages cités dans le texte.
\item Des conversions dynamiques vers le mot en langue philosophique depuis les éléments du thésaurus.
\end{enumerate}
\subsection{Les limites de la réédition}
