\chapter{Nature et histoire de la digitalisation}\label{HistDigit}
%\chapterprecis{Où l'on pose les bases théoriques sur lesquelles repose l'entreprise considérée dans le présent ouvrage.}
\section{Définition}\label{HistDigitDef}
\subsection{Digitalisation vs numérisation}
La digitalisation est à distinguer de la numérisation.
Dans les deux cas, il y a création d'un fichier électronique, mais l'information n'est pas la même : la numérisation est une photographie du texte, la digitalisation convertit l'information de l'original dans un format conservant l'information sémantique : quels sont les caractères composant le texte, quelle est sa hiérarchie interne.
\subsection{Différentes encodages}
En passant de l'imprimé au numérique, il s'agit de convertir les caractères du texte en entités lisibles par la machine à des fins de manipulation.
Aux débuts de l'ère informatique, 
\section{Emploi}\label{HistDigitEmp}
\subsection{Pour la sauvegarde du patrimoine littéraire}
\subsubsection{Projet Gutenberg}
Le 4 juillet 1971, Michael Hart pose la première pierre du projet Gutenberg en mettant en ligne la Déclaration de l'indépendance des États-Unis depuis les ordinateurs de l'université de l'Illinois.
À l'époque, les réseaux informatiques sont limités aux grandes universités américaines, et la saisie et la diffusion des textes sont relativement lentes ; mais Hart est vite rejoint par des volontaires, et le rythme de publication augmente d'année en année.
De dix textes en 1989, on passe à mille en 1997, puis à dix mille dès 2003 \parencite{lebert_projet_2010}.

Comme tous les textes sont mis à disposition librement sur des serveurs situés aux États-Unis, ils doivent respecter les lois américaines sur le copyright.
Du fait de la diffusion mondiale du projet, il peut entrer en conflit avec les lois locales sur le droit d'auteur : ainsi, en 2015, S. Fischer Verlag, un éditeur allemand, demanda le retrait de dix-huit titres de la bibliothèque en ligne que les lois locales protégeaient encore.
Un compromis fut trouvé en 2018 d'après lequel la fondation Gutenberg supprima l'accès à toutes les pages de son domaine aux adresses IP allemandes \parencite{noauthor_court_2018}.
\subsubsection{Google Books}
En 2002, Google commence à scanner 
\subsubsection{Internet Archive}
L'organisation à but non-lucrative Internet Archive, surtout connue pour son travail de sauvegarde des sites web (Wayback Machine) présente également un volet de stockage de numérisations, dont des livres.
En octobre 2019, près de 2500000 livres étaient présents sur le site, une moitié librement consultable par tous (livres hors-copyright), l'autre pouvant être lue en ligne par les inscrits au service (une personne à la fois peut consulter un livre\footnote{\url{https://archive.org/details/books} (consulté le 15 octobre 2019)}.

Internet Archive propose un service payant de numérisation des collections des institutions.
Avec 33 centre de scanneurs répartis dans le monde entier, l'organisation peut numériser mille pages par jour, qui sont transformées en PDF couleur de qualité haute permettant la recherche de texte (grâce à la reconnaissance optique des caractères)\footnote{\url{https://archive.org/scanning} (consulté le 15 octobre 2019)}.
Elle permet aussi aux utilisateurs enregistrés de téléverser leurs propres numérisations dans une collection open source \og Community Texts \fg\footnote{\url{https://archive.org/details/opensource} (consulté le 15 octobre 2019)}.
\subsubsection{Wikisource}
La fondation Wikimedia, qui chapeaute entre autres l'encyclopédie Wikipédia, lança en 2003 Wikisource pour stocker des textes libres de droits (domaine public ou œuvres libres).
Ils sont digitalisés, c'est à dire que le texte brut est mis en forme selon la syntaxe wiki.
Comme Wikisource est un projet collaboratif ouvert, il se pose la question de la validité des sources ; une extension du logiciel de Wikimedia permet de comparer le texte à une image numérisée servant de source\footnote{\url{https://en.wikipedia.org/wiki/Wikisource} (consulté le 15 octobre 2019}.
\subsubsection{Initiatives nationales}
À côté de ces projets d'envergure mondiale (quoique basés aux USA), certains pays fournissent leurs propres efforts pour numériser leurs fonds patrimoniaux.
Ainsi, la Bibliothèque nationale de France gère depuis 1997 le portail Gallica, mettant en ligne les images des livres des collections nationales accompagnés d'une notice descriptive\footnote{\url{https://gallica.bnf.fr}}.
La plateforme Europeana lancé en 2008 par la Commission européenne sert d'interface de recherche permettant d'accéder aux fonds numériques des institutions culturelles européennes partenaires, dont Gallica\footnote{\url{https://www.europeana.eu/portal/fr}}.
\subsection{Dans les humanités}
La digitalisation est très importante dans les humanités, où elle permet de sauvegarder des textes dont le support a vieilli, et de fournir un matériel de travail interactif aux chercheurs.
